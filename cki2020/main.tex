% !TeX program = xelatex

\documentclass{beamer}
\usetheme{metropolis}
\def\manjari{\fontspec[Script=Malayalam]{Manjari}}
% \setsansfont{Manjari}

\title{KDE Kerala}
\subtitle{The story of FOSS \& KDE in Kerala}
\author{Subin Siby}
\institute{subinsb.com/s/cki2020}
\date{2020-01-18}

\begin{document}
\begin{frame}
	\maketitle
\end{frame}

\begin{frame}
\frametitle{Outline}
\tableofcontents
\end{frame}

\section{History of FOSS In Kerala}
\begin{frame}
	\frametitle{Kerala}
	
	\begin{figure}
		\includegraphics[scale=0.15]{assets/kl}
		\center{\tiny \href{https://en.wikipedia.org/wiki/File:IN-KL.svg}{Made by Filpro, CC-BY-SA 4.0}}
	\end{figure}
	
	{\small \manjari
	Language: Malayalam (മലയാളം) \\
	Speakers: 37 Million
	}
\end{frame}
\begin{frame}
	\frametitle{History of FOSS In Kerala}
	\begin{itemize}
		\item TeX \\ 1980
		\item Slackware Linux \\ 1996
		\item River Valley Technologies \\ First Free Software based company in state
		\item \href{https://en.wikipedia.org/wiki/Free_Software_Users_Group,_Thiruvananthapuram}{Linux Users Group, Thiruvananthapuram} \\ 1998
		\item Freedom First! conf \\ 2001 \\ \href{https://www.gnu.org/press/2001-07-20-FSF-India.html}{FSF-India}
	\end{itemize}
\end{frame}

\begin{frame}
\frametitle{History of FOSS In Kerala}
\begin{itemize}
	\item FOSS.in Bangalore \\ 2001-2012
	\item \href{https://smc.org.in}{Swathanthra Malayalam Computing} \\ 2001
	\item IT@School \\ 2002
	\item FOSS In Textbooks \\ 2005
\end{itemize}
\href{https://swatantryam.blogspot.com/2007/08/story-of-free-software-in-kerala-india.html}{{\tiny Click here for more}}
\end{frame}

\begin{frame}
\frametitle{Swathanthra Malayalam Computing}
{\centering \manjari
"എന്റെ കമ്പ്യൂട്ടറിനു് എന്റെ ഭാഷ" \\
"My language for/on My Computer" \\
}
\vspace{5mm}
\begin{itemize}
	\item Traditional Malayalam - 1200+ glyphs
	\item Reformed Malayalam - 200+ glyphs
\end{itemize}
\begin{itemize}
	\item ASCII - 256
	\item Unicode - 1 Million+
\end{itemize}
\end{frame}

\begin{frame}
\frametitle{Swathanthra Malayalam Computing}
\begin{figure}
	\includegraphics[scale=0.8]{assets/ku}
	\center{\small \href{https://twitter.com/SubinSiby/status/1152236368409399297}{Traditional vs Reformed Malayalam}}
\end{figure}
\end{frame}

\begin{frame}
\frametitle{Swathanthra Malayalam Computing}
\begin{figure}
	\includegraphics[scale=0.1]{assets/smcl}
	\center{\tiny \href{https://smc.org.in}{SMC Logo}}
\end{figure}
\end{frame}

\section{FOSS In Education}
\begin{frame}
\frametitle{FOSS In Schools}
\begin{itemize}
	\item Gov. Schools from STD.1-12 are on FOSS
	\item IT Textbooks with FOSS syllabus
	\item \href{https://itschool.gov.in}{IT@School GNU/Linux OS}
	\item Lab exams
	\item \href{http://schoolsasthrolsavam.in}{IT Fair}
	\item \href{https://www.thehindu.com/news/national/kerala/making-physics-experiments-easier/article27190001.ece}{Practicals with Open Hardware}
\end{itemize}
\end{frame}

\begin{frame}
\frametitle{STD 1 IT Textbook}
\begin{figure}
	\includegraphics[scale=0.33]{assets/textbook}
	\center{\tiny \href{https://www.itschool.gov.in/2017/std1english.pdf}{Std1-textbook.pdf}}
\end{figure}
\end{frame}

\begin{frame}
\frametitle{IT@School GNU/Linux OS}
\begin{figure}
	\includegraphics[scale=0.33]{assets/cd}
	\center{\tiny \href{https://www.itschool.gov.in/Downloads/11picture6.jpg}{cd.jpg}}
\end{figure}
\end{frame}

\begin{frame}
\frametitle{IT@School GNU/Linux OS}
\begin{itemize}
	\item Ubuntu LTS base
	\item Mate DE
	\item Scratch
	\item KStars
	\item GeoGebra
	\item GCompris
	\item KTechLab
	\item Kalzium
	\item more...
\end{itemize}
\end{frame}

\begin{frame}
\frametitle{Little KITE Clubs}
\begin{figure}
	\includegraphics[scale=0.045]{assets/kite}
	\center{\tiny \href{https://kite.kerala.gov.in/}{KITE}}
\end{figure}
{\centering Local student clubs in schools to teach tech with FOSS tools}
\end{frame}

\begin{frame}
\frametitle{Little KITE Clubs}
\begin{itemize}
	\item Blender
	\item IoT
	\item Krita, GIMP
	\item School Wiki \\ \href{https://schoolwiki.in}{schoolwiki.in}
\end{itemize}
\end{frame}

\begin{frame}
\frametitle{IT Fair}
School -> District -> State level competitions.
\begin{itemize}
	\item Digital Painting (GIMP)
	\item Scratch Programming
	\item Animation
	\item Presentation
	\item Web Page designing
	\item Malayalam Typing
	\item IT Quiz
	\item ICT Teaching Aid
\end{itemize}
\end{frame}

\section{Localization}
\begin{frame}
	\frametitle{Localization}
	{\centering
		l10n \\
		Translating each strings from a master language to different languages
		\\
		\vspace{5mm}
		English -> Malayalam, Tamil
		\\
		Chinese -> English, Malayalam
		\\
	}
\end{frame}

\begin{frame}
\frametitle{Traditional Localization Process}
\begin{itemize}
	\item Pick a file
	\item Let the localization team know (Lock file)
	\item Download the PO file
	\item Edit, localize
	\item Send to the team
	\item Wait till review
	\item If corrections needed, goto step 1
	\item If everything's alright,
	\item Commit file
	\item Remove the lock
\end{itemize}
\end{frame}

\begin{frame}
\frametitle{PO File}
\begin{figure}
	\includegraphics[scale=0.5]{assets/po}
	\center{\small Part of Dolphin's PO file}
\end{figure}
\end{frame}

\begin{frame}
\frametitle{Problems with Traditional Localization}
\begin{itemize}
	\item Difficult for a new contributor
	\item Workload on maintainer
	\item Simultaneous localization not possible
	\item Translation Memory
	\item Difficult local setup
\end{itemize}
\end{frame}

\begin{frame}
\frametitle{Localization Systems}
\begin{itemize}
	\item KDE \\ Traditional Localization \\ \href{https://l10n.kde.org}{l10n.kde.org}
	\item GNOME \\ Damned Lies! \\ \href{https://l10n.gnome.org}{l10n.gnome.org}
	\item F-Droid \\ Weblate \\ \href{https://hosted.weblate.org}{hosted.weblate.org}
	\item Firefox \\ Mozilla Pontoon \\ \href{https://pontoon.mozilla.org/}{pontoon.mozilla.org}
	\item Calamares \\ Transifex
\end{itemize}
\end{frame}

\begin{frame}
\frametitle{KDE Malayalam Localization}
\begin{itemize}
	\item kde.smc.org.in
	\item Weblate
	\item Suggestions
	\item Voting
\end{itemize}
\end{frame}

\begin{frame}
\frametitle{Localization Data For Machine Translation}
\begin{figure}
	\includegraphics[scale=0.23]{assets/translation}
	\center{\small \href{http://opusmt.wmflabs.org/}{opusmt.wmflabs.org}}
\end{figure}
\end{frame}

\begin{frame}
\frametitle{KDE Malayalam Localization Team}
\begin{figure}
	\includegraphics[scale=0.23]{assets/ml-team}
	\center{\small \href{http://wiki.smc.org.in/kde}{KDE Malayalam Kochi Sprint}}
\end{figure}
\end{frame}

\begin{frame}
\frametitle{KDE Malayalam Localization Team}
\begin{figure}
	\includegraphics[scale=0.23]{assets/ml-team2}
	\center{\small \href{http://wiki.smc.org.in/kde}{KDE Malayalam Thrissur Sprint}}
\end{figure}
\end{frame}

\begin{frame}
\frametitle{Our College FOSS Club}
\begin{figure}
	\includegraphics[scale=0.23]{assets/fossers}
	\center{\small \href{http://fossers.vidyaacademy.ac.in}{FOSSers}}
\end{figure}
\end{frame}

\begin{frame}
\frametitle{Free Software Community of India}
\begin{figure}
	\includegraphics[scale=0.23]{assets/fsci}
	\center{\small \href{http://fsci.in}{fsci.in} meetup at Thrissur}
\end{figure}
\end{frame}

\begin{frame}
\frametitle{Questions ?}
{\centering
	Subin Siby \\
	/bin/su \\
	subinsb.com/s/cki2020 \\ \vspace{5mm}
	Mastodon: \href{https://aana.site/subins2000}{@subins2000@aana.site} \\
	Twitter: \href{https://twitter.com/SubinSiby}{@SubinSiby} \\
	Instagram: \href{https://instagram.com/subins2000/}{@subins2000} \\
}
\end{frame}

\end{document}